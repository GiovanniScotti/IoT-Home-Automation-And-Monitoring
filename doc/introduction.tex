\section{Purpose}
This document is intended to describe a \textit{Home Automation and Monitoring} system, its electronic components as well as its constraints and the interaction with the real world and the end user. A useful example of work environment will be provided to clarify the scope of the system itself and how it can be deployed.

The documentation is full of diagrams, pictures and schematics to let the final user easily implement the infrastructure and take advantage of its countless features.

This document is mainly addressed to computer scientists, IoT enthusiasts, makers and anyone, with some electronics and programming skills, who wants to keep an eye on his/her home and to control devices remotely.

\section{Scope}
The \textit{Home Automation and Monitoring} system aims to offer a smart solution to home automation and monitoring needs. It is intended for those kind of users who want to visualize information about their home, such as temperature or humidity, and control devices remotely.

The system consists mainly of:
\begin{itemize}
	\item a \textit{back-end}, which is in charge of managing MQTT messaging protocol and runs Node-RED. Data coming from boards provided with sensors and actuators are gathered and sent to a remote service provider to store information into the cloud and to open up further analysis.
	
	\item a \textit{front end}, which is offered by a user-friendly and web-accessible Node-RED dashboard that can be easily reached from any device either connected to the same network or to the Internet by means of port forwarding.
\end{itemize}

\noindent
The system must be secure. This is why authentication is required in order to access the dashboard.
Moreover, MQTT must be secured too and clients provide credentials to the broker to join the network.
Last but not least, power consumption has been taken into account: battery-powered boards can enter a deep sleep state after publishing data.

\section{Definitions, Acronyms, Abbreviations}

\begin{itemize}
	\item \textbf{ADC:} Analog to Digital Converter.
	\item \textbf{AP:} wireless Access Point, it allows Wi-Fi devices to connect to a wired network. Usually it is an integral component of routers for home or office use.
	\item \textbf{Back-end:} any device or computer program that remains in the background and offers application logic and communication interfaces to work with the front-end counterpart. It can provide a data access layer.
	\item \textbf{Front-end:} any part of a system the users directly interact with. It provides the so called presentation layer.
	\item \textbf{IoT:} Internet Of Things.
	\item \textbf{LAN:} Local Area Network.
	\item \textbf{MQTT:} Message Queuing Telemetry Transport, a publish-subscribe messaging protocol.
	\item \textbf{System:} the entire infrastructure and all the devices involved in the \textit{Home Automation and Monitoring} project.
	\item \textbf{SoC:} System-On-Chip, an integrated circuit that contains various electronic components designed to work together to achieve a common goal.
	\item \textbf{UI}: User Interface.
\end{itemize}